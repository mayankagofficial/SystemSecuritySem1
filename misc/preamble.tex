\setcounter{tocdepth}{3} % Table of contents includes sections up to: Subsubsection
\setcounter{secnumdepth}{3} % Sections are numbered up to: Subsubsection

% Load packages
\usepackage[utf8]{inputenc} % support of UTF-8 input text
\usepackage[T1]{fontenc} % proper encoding of the output font
\usepackage{lmodern} % original CM font doesn't work well with [T1]{fontenc}
% use different sans font
\usepackage[semibold]{sourcesanspro}
\usepackage{microtype}
\usepackage[all, defaultlines=2]{nowidow}

% set language
\ifenglish%
    \usepackage[english]{babel}
\else
    \usepackage[ngerman]{babel}
\fi

% utility packages
\usepackage{ifpdf} % if else structure that checks if a PDF is compiled
\usepackage{ifthen} % adds support for if else structures

% defining the UPB-colors
\usepackage{xcolor}
\definecolor{upbblue}{RGB}{0,62,121}
\definecolor{upbdarkgray}{RGB}{45,45,45}
\definecolor{upbgray}{RGB}{85,85,85}
\definecolor{upblightgray}{RGB}{170,170,170}
\definecolor{upbred}{RGB}{165,1,53}
\definecolor{upbgreen}{RGB}{134,166,6}
\definecolor{upbdarkgreen}{RGB}{94,126,20}
\definecolor{upbcyan}{RGB}{4,156,206}
\definecolor{upborange}{RGB}{202,109,0}
\definecolor{upbcassis}{RGB}{169,57,131}
\definecolor{upblightblue}{RGB}{0,127,185}

%% Floats
\usepackage{caption} % captions for floats
\usepackage{standalone} % allows file with standalone class to be inputted


% Figures
\usepackage{graphicx} % adds functionality to \includegraphics
\usepackage{subfig} % provides the \subfigure command
\usepackage{placeins} % provides \FloatBarrier command
\graphicspath{{images/}{figures/}} % so you don't have to provide the path for every image

% Tables
\usepackage{booktabs} % adds rules for tables in a formal style
\usepackage{multirow} % add support for multi-rows
\usepackage{multicol} % add support for multi-columns
\usepackage{longtable} % long tables can break pages
\usepackage{array} % provides >{} and <{} syntax for column definition
% column types
% (l)eft, (c)enter, (r)ight aligned
% p{<length>} left alined with defined width (paragraph)
% X left aligned with max possible width
\usepackage{tabularx}
% Y centered with max possible width
\newcolumntype{Y}{>{\centering\arraybackslash\hspace{0pt}}X}
% Z right aligned with max possible width
\newcolumntype{Z}{>{\raggedleft\arraybackslash}X}
% x alias of p for consistency
\newcolumntype{x}[1]{p{#1}}
% y{<length>} centered with defined width
\newcolumntype{y}[1]{>{\centering\arraybackslash}p{#1}}
% z{<length>} centered with defined width
\newcolumntype{z}[1]{>{\raggedleft\arraybackslash}p{#1}}
\usepackage[flushleft]{threeparttable} % text below tables

% Math stuff and units
\usepackage{latexsym, amsmath, amssymb, amsfonts, upgreek}
\usepackage{siunitx}
\newcommand{\mathup}{\mathrm}

% Glossary
\usepackage[nonumberlist, acronym, toc]{glossaries}

% Enable quotes by \enquote{}
\usepackage[autostyle, english=american, german=quotes]{csquotes}

% Allows to create automata and fancy graphics
\usepackage{tikz}

% Source code listings
\newcommand{\code}[1]{\texttt{#1}}
\usepackage{listings}
\lstset{
    floatplacement=htp,
    basicstyle=\ttfamily\footnotesize,
    identifierstyle=\color{upbblue},
    keywordstyle=\color{upborange},
    stringstyle=\color{upbdarkgreen},
    commentstyle=\color{upbgray},
    columns=flexible,
    tabsize=4,
    aboveskip={1.5\baselineskip},
    frame=tb,
    framerule=\heavyrulewidth,
    extendedchars=true,
    showspaces=false,
    showstringspaces=false,
    breaklines=true,
    backgroundcolor=,
    breakautoindent=true,
    captionpos=b,
    numbers=left,
    numberstyle=\tiny,
    numbersep=5pt,
}


\usepackage[ruled, vlined, linesnumbered, algo2e]{algorithm2e}

% use some standards for mathematical expressions:
\newcommand{\red}{{\rm red}}
\newtheorem{theorem}{Theorem}[section]
\newtheorem{lemma}[theorem]{Lemma}
\newtheorem{proposition}[theorem]{Proposition}
\newtheorem{corollary}[theorem]{Corollary}
\newtheorem{algorithm}[theorem]{Algorithm}
\newenvironment{example}{\begin{quote}{\bf Example:}}{\end{quote}}
% \newtheorem{definition}[theorem]{Definition}
% gray definition boxes, that way you'll find them in the text
\usepackage{shadethm}
\newshadetheorem{sthm}[figure]{Definition}
\newenvironment{definition}[1][]{%
    \definecolor{shadethmcolor}{rgb}{.9,.9,.9}%
    \begin{sthm}[#1]%
}{%
    \end{sthm}%
}

%% type setting
\usepackage[defaultlines=2,all]{nowidow} % prevents widows and orphans (single lines at beginning/end of page) enforces minimum of <defaultline> instead
\usepackage{microtype} % improves typesetting
\UseMicrotypeSet[protrusion]{basicmath} % disable protrusion for tt fonts

% align footnotes
\usepackage[hang,marginal]{footmisc}
\def\footnotemargin{1em}

% smaller dot for itemize
\renewcommand\labelitemi{$\vcenter{\hbox{\tiny$\bullet$}}$}

% define paragraphs to have a dot at the end
\let\origparagraph\paragraph%
\renewcommand{\paragraph}[1]{\origparagraph{#1.}}

% apparently some packages used here utilize deprecated KOMA-Script functionalities
% scrhack ensures, that those packages don't break suddenly and removes the deprecation warnings from the log
\usepackage{scrhack}

% bibliography
% we don't use biber, because we it is very slow when parsing cryptobib and rfcbib
\usepackage[backend=bibtex]{biblatex}
\addbibresource{literature/literature.bib}
\addbibresource{literature/cryptobib/crypto.bib}
\addbibresource{literature/rfcbib/data/rfc.bib}

% allow urls to break at hyphens -
\usepackage{url}
\def\UrlBreaks{\do\/\do-}


% Hyperlinks and menu for your document
\usepackage[%
    breaklinks,%
    hyperindex,%
    colorlinks,%
    anchorcolor=black,%
    citecolor=black,%
    filecolor=black,%
    linkcolor=black,%
    menucolor=black,%
    urlcolor=black,%
    pdftex%
]{hyperref} % pagebackref: Add page number to the references where they can be found

% set autoref names
\addto\extrasenglish{%
\renewcommand{\chapterautorefname}{Chapter}%
\renewcommand{\sectionautorefname}{Section}%
\renewcommand{\subsectionautorefname}{Section}%
\renewcommand{\subsubsectionautorefname}{Section}%
\renewcommand{\paragraphautorefname}{Paragraph}%
\renewcommand{\subparagraphautorefname}{Paragraph}%
\renewcommand{\appendixautorefname}{Appendix}%
}

% example of a custom listing language definition
\lstdefinelanguage{XML}{
    morestring=[b]",
    moredelim=[s][\bfseries\color{upbblue}]{<}{\ },
    moredelim=[s][\bfseries\color{upbblue}]{</}{>},
    moredelim=[l][\bfseries\color{upbblue}]{/>},
    moredelim=[l][\bfseries\color{upbblue}]{>},
    morecomment=[s]{<?}{?>},
    morecomment=[s]{<!--}{-->},
    alsoletter={-:},
    identifierstyle=\color{upbdarkgreen},
    keywordstyle=\color{upbblue},
    stringstyle=\color{upborange},
    commentstyle=\color{upbgray},
}

\usepackage{todonotes} % provides the \todo command

\usepackage{pifont} % defines some symbols http://ctan.org/pkg/pifont
\newcommand*{\cmark}{\ding{51}}% checkmark
\newcommand*{\xmark}{\ding{53}}% crossmark


% DO NOT LOAD ANY OF YOUR PACKAGES BEYOND THIS PACKAGE

\makeatletter
\AtBeginDocument{
    \hypersetup{
        pdftitle = {\@title},
        pdfauthor = {\@author},
        pdfsubject={\@title},
    }
}
\makeatother

% Use the same counter for tables and figures
\makeatletter
\AtBeginDocument{
\let\c@table\c@figure
\let\c@lstlisting\c@table
\let\c@algocf\c@lstlisting
}
\makeatother

\ifpdf
    \hypersetup{linktocpage=false} % false=links are section names, true=links are page numbers, IMPORTANT: in dvi2ps mode, 'true' is required!
\else
    \hypersetup{linktocpage=true} % false=links are section names, true=links are page numbers, IMPORTANT: in dvi2ps mode, 'true' is required!
    \usepackage[hyphenbreaks]{breakurl}
\fi
